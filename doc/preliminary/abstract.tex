\begin{abstract}
    

\begin{center}
    Shoney Arickathil, School of Informatics, SRH Hochschule Heidelberg \\

Master Thesis \\
\textbf{Application of Natural language processing in ECommerce}\\
\textbf{Predicting product taxonomy}

\end{center}


Electronic commerce started in early 1990's in which the business transactions are conducted through computer networks. The ease of buying a product online has reduced physical work and time required for decision-making by compare the product features. A wide range of products are sold online. One of the challenge is to well define a products' taxonomy. A product taxonomy is a hierarchical structure to organize the products in an E-commerce platform in such a way that customers can find the product in the fewest possible clicks.
Product taxonomy for the same product may differ based on the supplier and manufacturer (sources). Hence, product taxonomy details cannot be imported directly from the sources into the E-commerce platform. Product details cannot be live on E-commerce platform until its taxonomy is well-defined, this delays the time of product availability for customers. 

In this thesis, the author developed a machine learning classification model to predict the product taxonomy based on the features of the product. Author created a prototype of predicting the leaf node level of product category. This prototype formed as a foundation to create a machine learning model to predict the complete product taxonomy. Author describes the process of reducing dimensionality, text normalization, imputation of missing text.

In this paper, author developed a supervised \acl{RNN} for text classification. Author also describes the mathematical foundations behind the machine learning process. Author evaluates the model with confusion matrix.

\end{abstract}

