\chapter{Training a classification model}

One of the major challenge in an ecommerce industry is to categorize the products. The phenominal types of products in the ecommerce web application sold online may require an artificial inteligence generated category tree. The multi level product-categories in the taxonomy tree received and defined from the suppliers or manfucture may not be usable. Since the existing multi level category of those products in ecommerse application may defer. Importing product-category details directly from the various channels may lead to disambugution. The artificial inteligence generated category tree reduces the product ready to deploy time on the production environment. The product ready to deploy time here refers to the check lists of data correctness of the product before listing online.

\section {Fetch existing product taxonomy using Elastic Search}

In section \ref {pyenv}, version specific python client installation detail are documented. For this project, python client elasticsearch 6.8.2 is installed as the client needs to be compatble with Elastic search version being used.


\begin{lstlisting}[language=Python]
from elasticsearch import Elasticsearch
client = Elasticsearch("http://localhost:9200")

resp = client.search(index="development_products",
                     body={"_source":["descriptions","descriptionsSource","nameSource","shortDescriptionSource","categoriesSource"],
                           "query": {"match_all": {}}})
\end{lstlisting}

The above code sample fetches the features of the product such as "name","description",category".

\section {Normalization of text.}