\chapter{Conclusion and Outlook}


\section{Answer to Research Questions}
\begin{enumerate}[label=\textbf{RQ\arabic*:}]
    \item What are the use cases in which natural language processing and machine learning model can be utilized in an E-Commerce industry?
    
    Machine learning model can generate well-defined product taxonomy. Author introduces a methodology for such a model in section \ref{sec:ideate}. A well-defined product taxonomy is the foundation for many use cases like the ones listed below:
        \begin{itemize}
        \item Product recommendation system.
        
        The machine learning model can classify and recommend products from the wide range of products to the customer based on the customer's transaction history. The transaction history containing product relevance feedback such as click ``clicks'', ``cart-adds'', ``orders'', ``revenue'' \parencite{KarmakerSantu.2017} are usable if the taxonomy is well-defined. 
        
        \item Search bar autocompletion.
        
        A lot of research has been conducted in the field of user experience in terms of product search. A facet in user interface is an extended drop down menu with filtered result set. \parencite{Tagliabue.26052020} proposes a real time personalized catalog generation by combining customer search logs and in-session data. 
        
        \item Customer conversational shopping.
            
        \Parencite{ChrisMessina} states that gradually traditional commerce will shift towards conversational commerce. An experience of shopping on the go or hands-free assistance could be trending. Major Ecommerce  industries offering voice assistance devices includes feature of buying products on voice command.  


        \item Building a knowledge graph based on the textual details of a product. Refer section \ref{sec:building-kg}
        \item Virtual customer support chatbot may leverage the knowledge graph to answer queries raised by customers.
    \end{itemize}
    
    \item For a product classification task, how to define product features as an input for machine learning model?
    
   Based on the level of product taxonomy the number of features may vary. Table \ref{table:feature_decription} lists some features are listed simply by prior knowledge of the product domain. However, section \ref{sec:feature-selection} describes various methods to reduce dimensionality.

    
    \item How does a machine learn pattern in product features for classification?
    

    \begin{enumerate}[label=\textbf{SRQ\arabic*:}]
        \item What are the mathematical equations behind the learning process?
    
        Author has dedicated Chapter \ref{ch:math-behid} for step by step illustration of learning process. Author begins the chapter with basic mathematical representation of a neural network, followed by equations for feedforward, back-propagation. Further, author describes the probability distribution and its relation with the project in this paper. What is the role of partial derivatives to reduce the loss function is described.  
        \item What are the mathematical reasons for applying certain pytorch functions during the training process? 
        
        Section \ref{sec:Logsoft} describes mathematical reasons behind using Pytorch's functions such as LogSoftMax. Chapter \ref{ch:math-behid} is detailed with the explanation of applying log to a value.         

    \end{enumerate}
    
    \item What is the algorithm to train the machine learning model to predict the product taxonomy?
        
    Author modified the algorithm of \parencite{sean}   classifying names with character-level RNN. Instead of series of tensor representation of characters, author use tensor representation of features. The RNN model learns the pattern and predicts the product taxonomy.

  \end{enumerate}

  \section{Future scope}

  In this paper, author has researched on the machine learning task to text based classification. Research has been conducted on use cases of text based classification in E-Commerce. Author limited the research surrounding basic understanding of neural network, its mathematical equations, creating a model which predicts product taxonomy.  This paper is also a guide for individual who seek basic understanding of machine learning as well as overview of product taxonomy and its importance in E-commerce.

  There is a scope of research on image classification and its use cases in the E-commerce industry. Images are vital component of an E-commerce website. There is saying ``A picture is worth a thousand words''. The product image itself has so much information  to be tapped. 

  Few of them are listed below:
  \begin{itemize}
    \item Image captioning.
    
    Generating text from the image is a useful process in E-commerce. For example, product image also describes its features such as color. Such information can be added to the caption of the image. \parencite[Section 15.4.7]{pml1Book} describes an example of image captioning using the sequential data processing network with \textbf{attention algorithm}.
    
    \item Attention based image classification.
    
    \parencite{Vaswani.12062017} proposes a network architecture named Transformers which uses attention in the encoder and decoder in the machine translation tasks. Using the transformer model to process the sequence of pixels in a product image to identify the clustered components or to identify which other product could be a part of the product.
    For example, image from Appendix B figure \ref{fig:sparepart} is product image, using attention based image classification, identifying the products which fit in and around the processed product image could enable E-Commerce industries cluster the products.
 


    \item Image based categorization.
    
    A Machine learning model can generate product taxonomy based on the image of the product. Clustering the products based on the image could also enable to define a better product taxonomy. 

    \item Image to image translation.
   
    Using \acfp{GAN} \parencite{Goodfellow.31122016}, a sketch can be converted to photorealistic image.     
    

  \end{itemize}