\begin{abstract}
    

\begin{flushright}
   
    \justifying
    This research focuses on the application of machine learning in optimizing public transport planning by developing models to predict traffic flow at junction levels in various conditions, particularly on freeways or highways.

    The escalating challenge of urban traffic congestion underscores the critical need for a robust traffic monitoring and forecasting system, with accurate prediction of traffic flow being a core component. This study emphasizes the effectiveness of a proposed framework in understanding traffic flow waves and congestion at the junction level.

    The findings highlight the necessity of identifying the most effective solution among Recurrent Neural Network (RNN) variants, including Long Short-Term Memory (LSTM) models, neural networks. These models exhibit superior performance in link flow predictions compared to other machine learning methods. The choice between RNN, or LSTM models is crucial for enhancing the accuracy and efficiency of traffic flow predictions.

    Practical insights underscore the importance of the deep learning model structure, data pre-processing, and error matrices for achieving precise traffic flow predictions. Implementing the suggested approaches contributes to the optimization of traffic flow, thereby enhancing overall urban transportation efficiency and fostering a more sustainable and seamless urban mobility experience.
\end{flushright}
\end{abstract}

